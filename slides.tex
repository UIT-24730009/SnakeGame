\documentclass{beamer}
\usepackage[utf8]{vietnam}  % Hỗ trợ tiếng Việt
\usepackage{listings} % Gói để hiển thị mã nguồn
\usetheme{Madrid}
\title{Báo cáo Đồ án: Snake Game}
\author{Nhóm The Fellowship of the Mind}
\date{\today}

\begin{document}

% Title slide
\begin{frame}
    \titlepage
\end{frame}

% Table of contents
\begin{frame}{Mục Lục}
    \tableofcontents
\end{frame}

% Section: Giới thiệu
\section{Giới thiệu}
\begin{frame}{Tổng quan dự án}
    \begin{itemize}
        \item Trò chơi \textbf{Snake Game} - trò chơi hành động cổ điển.
        \item Mục tiêu: Điều khiển rắn để ăn thức ăn và tránh va chạm.
        \item Đồ án được phát triển với hai phiên bản:
        \begin{itemize}
            \item \textbf{V1:} Sử dụng C++.
            \item \textbf{V2:} Sử dụng Plain JavaScript và HTML.
        \end{itemize}
    \end{itemize}
\end{frame}

% Section: Tài liệu kỹ thuật
\section{Tài liệu kỹ thuật}
\begin{frame}{Cấu trúc trò chơi (V1 - C++)}
    \begin{itemize}
        \item \textbf{CSnake:} Quản lý trạng thái rắn và các phương thức:
        \begin{itemize}
            \item \texttt{initSnake()} - Khởi tạo rắn ban đầu.
            \item \texttt{updateSnake()} - Cập nhật vị trí và kiểm tra va chạm.
            \item \texttt{drawBackground()} - Hiển thị rắn và bảng chơi.
        \end{itemize}
        \item \textbf{Main Loop:} Điều khiển logic chơi qua vòng lặp chính:
        \begin{itemize}
            \item Kiểm tra đầu vào.
            \item Cập nhật trạng thái rắn.
            \item Tạo thức ăn mới khi cần.
        \end{itemize}
    \end{itemize}
\end{frame}

\begin{frame}{Cấu trúc trò chơi (V2 - JavaScript)}
    \begin{itemize}
        \item \textbf{Snake:} Quản lý trạng thái và logic di chuyển của rắn.
        \item \textbf{Food:} Sinh ngẫu nhiên vị trí thức ăn.
        \item \textbf{Game:} Quản lý toàn bộ vòng đời trò chơi, bao gồm:
        \begin{itemize}
            \item Vẽ màn hình.
            \item Cập nhật điểm.
            \item Kiểm tra va chạm.
        \end{itemize}
    \end{itemize}
\end{frame}

\begin{frame}[fragile]{Đoạn mã quan trọng (V2)}
    \textbf{Cập nhật trạng thái của rắn:}
    \begin{lstlisting}[language=JavaScript]
update() {
    const head = {
        x: this.body[0].x + this.direction.x,
        y: this.body[0].y + this.direction.y
    };

    if (head.x < 0) head.x = this.gridSize - 1;
    if (head.x >= this.gridSize) head.x = 0;
    if (head.y < 0) head.y = this.gridSize - 1;
    if (head.y >= this.gridSize) head.y = 0;

    this.body.unshift(head);
    if (!this.shouldGrow) this.body.pop();
    else this.shouldGrow = false;
}
    \end{lstlisting}
\end{frame}

% Section: Kỹ năng đã áp dụng
\section{Kỹ năng đã áp dụng}
\begin{frame}{Kỹ năng đã áp dụng}
    \begin{itemize}
        \item \textbf{Kỹ năng lập trình:}
        \begin{itemize}
            \item Phát triển logic game với C++ và JavaScript.
            \item Thiết kế giao diện đơn giản nhưng hiệu quả.
        \end{itemize}
        \item \textbf{Kỹ năng làm việc nhóm:}
        \begin{itemize}
            \item Sử dụng Trello và Slack để phối hợp.
            \item Quản lý mã nguồn và giải quyết xung đột trên GitHub.
        \end{itemize}
        \item \textbf{Kỹ năng giải quyết vấn đề:}
        \begin{itemize}
            \item Tìm giải pháp tối ưu để cải thiện hiệu suất.
            \item Sửa lỗi phát sinh và cải thiện trải nghiệm người dùng.
        \end{itemize}
    \end{itemize}
\end{frame}

% Section: Kết quả đạt được
\section{Kết quả đạt được}
\begin{frame}{Kết quả đạt được}
    \begin{itemize}
        \item \textbf{Hoàn thành đúng hạn:}
        \begin{itemize}
            \item Hai phiên bản game chạy ổn định.
        \end{itemize}
        \item \textbf{Tài liệu kỹ thuật:}
        \begin{itemize}
            \item Cung cấp báo cáo chi tiết và chuyên sâu.
        \end{itemize}
        \item \textbf{Phát triển kỹ năng cá nhân:}
        \begin{itemize}
            \item Cải thiện kỹ năng lập trình và làm việc nhóm.
        \end{itemize}
    \end{itemize}
\end{frame}

% Section: Hướng mở rộng
\section{Hướng mở rộng}
\begin{frame}{Hướng mở rộng}
    \begin{itemize}
        \item \textbf{Cải tiến giao diện:}
        \begin{itemize}
            \item Thêm animations và âm thanh.
        \end{itemize}
        \item \textbf{Tính năng mới:}
        \begin{itemize}
            \item Chế độ chơi nhiều người.
            \item Tăng độ khó theo thời gian.
        \end{itemize}
        \item \textbf{Tích hợp công nghệ mới:}
        \begin{itemize}
            \item Sử dụng React và Phaser cho phiên bản web.
        \end{itemize}
    \end{itemize}
\end{frame}

% Section: Kết luận
\section{Kết luận}
\begin{frame}{Tổng kết}
    \begin{itemize}
        \item Hoàn thành mục tiêu dự án: Trò chơi hoạt động tốt, ổn định.
        \item Rèn luyện kỹ năng làm việc nhóm và lập trình.
        \item Đặt nền tảng cho các dự án phức tạp hơn trong tương lai.
    \end{itemize}
\end{frame}

% Thank you slide
\begin{frame}{Cảm ơn}
    \centering
    \Huge Cảm ơn quý thầy cô và các bạn đã lắng nghe!
\end{frame}

\end{document}
