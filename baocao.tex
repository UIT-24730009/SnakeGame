\documentclass[a4paper, 12pt]{article}
\usepackage{fullpage}
\usepackage{titlesec} % Gói để tùy chỉnh tiêu đề
\usepackage[utf8]{vietnam}  % Hỗ trợ tiếng Việt
\usepackage{enumitem}
\usepackage{xcolor}  % Hỗ trợ màu sắc
\usepackage{hyperref}  % Hỗ trợ tạo link
\usepackage{graphicx}
\hypersetup{
    colorlinks=true,
    linkcolor=blue,
    filecolor=blue,
    urlcolor=blue,
}
\usepackage{array}
\usepackage{listings} % Gói để hiển thị mã nguồn
\usepackage{xcolor} % Để tô màu cho mã nguồn
\lstset{
    language=C++, % Ngôn ngữ lập trình (có thể thay bằng Java, C++, etc.)
    basicstyle=\ttfamily\small, % Font chữ cơ bản
    keywordstyle=\color{blue}\bfseries, % Màu sắc cho từ khóa
    commentstyle=\color{green!50!black}, % Màu sắc cho chú thích
    stringstyle=\color{red}, % Màu sắc cho chuỗi
    numbers=left, % Đánh số dòng ở bên trái
    numberstyle=\tiny\color{gray}, % Kiểu số dòng
    stepnumber=1, % Hiển thị số dòng cách nhau 1 dòng
    breaklines=true, % Tự động ngắt dòng
    frame=single, % Khung bao quanh mã nguồn
}

% \title{}
% \author{}
% \date{}

\begin{document}
% \maketitle

% Bia bao cao
\thispagestyle{empty}
\begin{center}
    \begin{center}
        \textbf{ĐẠI HỌC QUỐC GIA TP. HỒ CHÍ MINH}\\[0.25cm]
        \large \textbf{TRƯỜNG ĐẠI HỌC CÔNG NGHỆ THÔNG TIN}\\[0.25cm]
        \large \textbf{KHOA KHOA HỌC VÀ KỸ THUẬT THÔNG TIN}
    \end{center}
    \vspace{1cm}

    \begin{figure}[h]
        \centering
        \includegraphics[width=0.2\textwidth]{uit.png} % adjust width as needed
        % \caption{Your image caption here}
        % \label{fig:your-label}
    \end{figure}
    \vspace{1cm}
    
    \begin{center}
        \Large \textbf{BÁO CÁO ĐỒ ÁN}\\[0.5cm]
        \Large \textbf{MÔN HỌC: KỸ NĂNG NGHỀ NGHIỆP}\\[0.25cm]
        \large \textbf{LỚP: SS004.E11.CN1}
    \end{center}
    \vspace{1cm}
    
    \begin{center}
        \Large \textbf{ĐỀ TÀI}\\[0.5cm]
        \huge \textbf{SNAKE GAME}
    \end{center}
    \vspace{1cm}
    
    \begin{table}[h]
        \centering
        \begin{tabular}{l c}
            \multicolumn{2}{l}{Giảng viên hướng dẫn: ThS. Nguyễn Văn Toàn} \\[0.5cm]
            \multicolumn{2}{l}{Sinh viên thực hiện:} \\[0.25cm]
    
            \multicolumn{2}{c}{
                \begin{minipage}{0.5\textwidth}
                    \centering
                    \begin{tabular}{l @{\hspace{1cm}} c}
                        Trần Mai Uyên Nhi & 24730053 \\
                        Phạm Tiến Anh & 24730009 \\
                        Nguyễn Thanh Minh & 24730046 \\
                        Lưu Đinh Đại Đức & 24730022 \\
                        Trương Anh Vũ & 23730232 \\
                    \end{tabular}
                \end{minipage}
            } \\
        \end{tabular}
        % \caption{Caption}
        \label{tab:my_label}
    \end{table}
    \vspace{1cm}
    
    \textbf{TP. Hồ Chí Minh, tháng 11 năm 2024}
    \vspace{1cm}
\end{center}
\newpage

\renewcommand{\contentsname}{\hfill \textbf{Mục lục} \hfill} % Căn giữa tiêu đề mục lục
\renewcommand\thesection{\Roman{section}.}
\renewcommand\thesubsection{\arabic{subsection}.}
\renewcommand\thesubsubsection{\thesubsection\arabic{subsubsection}}
\tableofcontents

\newpage
% Noi dung bao cao
% Task 1
\section{Hợp đồng nhóm}
\subsection{Thông tin}
\renewcommand{\arraystretch}{1.5}
\begin{itemize}
    \item Tên nhóm: \large \textbf{The Fellowship of the Mind}
    \item Thành viên:
    \begin{table}[h]
        \centering
        \begin{tabular}{|c|>{\raggedright\arraybackslash}p{5cm}|c|}
            \hline
            \textbf{STT} & \multicolumn{1}{c|}{\textbf{Tên}} & \textbf{Mã số sinh viên} \\
            \hline
            1 & Trần Mai Uyên Nhi & 24730053 \\
            \hline
            2 & Phạm Tiến Anh & 24730009 \\
            \hline
            3 & Nguyễn Thanh Minh & 24730046 \\
            \hline
            4 & Lưu Đinh Đại Đức & 24730022 \\
            \hline
            5 & Trương Anh Vũ & 23730232 \\
            \hline
        \end{tabular}
        % \caption{Caption}
        % \label{tab:my_label}
\end{table}
\end{itemize}

\subsection{Mục tiêu}
\begin{itemize}
    \item Cùng nhau giúp đỡ, động viên, hoàn thành tốt, đạt kết quả cao trong môn học.
    \item Rèn luyện kỹ năng làm việc nhóm, kỹ năng giao tiếp, và kỹ năng trình bày.
    \item Cải thiện và nâng cao kỹ năng logic, kỹ năng giải quyết vấn đề của môn học.
\end{itemize}

\subsection{Vai trò của các thành viên}
\begin{itemize}
    \item \textbf{Trần Mai Uyên Nhi} - Thảo luận định hướng dự án, theo dõi tiến độ công việc của các thành viên.
    \item \textbf{Phạm Tiến Anh (Trưởng nhóm)} - Tạo báo cáo, khái quát yêu cầu cơ bản, lập kế hoạch công việc, hỗ trợ lập trình game logic.
    \item \textbf{Nguyễn Thanh Minh} - nghiên cứu tech stack, khởi tạo code base, khởi tạo assets, animations, lập trình game logic.
    \item \textbf{Lưu Đinh Đại Đức} - tìm hiểu tech stack, lập trình game logic.
    \item \textbf{Trương Anh Vũ} - Tạo báo cáo, tìm hiểu tech stack, lập trình game logic.
\end{itemize}

\subsection{Hiệp định}
\begin{itemize}
    \item Nhóm đưa ra quyết định dựa trên kiến thức nền tảng của các thành viên để chọn tech stack phù hợp.
    \item Ý kiến được chọn nếu 4/5 thành viên đồng ý.
    \item Trong trường hợp có tranh cãi, trưởng nhóm sẽ có quyền quyết định cuối cùng.
\end{itemize}

% Task 2
\section{Công cụ và không gian làm việc}
\begin{itemize}
    \item \href{https://github.com/UIT-24730009/SnakeGame}{\large \textbf{Github repository}} - Công cụ làm việc nhóm.
    \item \href{https://app.slack.com/client/T07Q56DLLUX/C07U74U2XGF}{\large \textbf{Slack}} - Công cụ giao tiếp.
    \item \href{https://www.overleaf.com/project/67271c85e33c6e0dfe041c9d}{\large \textbf{Overleaf}} - Công cụ soạn thảo văn bản.
    \item \href{https://github.com/users/UIT-24730009/projects/1/views/1}{\large \textbf{Github Projects}} - Công cụ quản lý, theo dõi tiến độ công việc.
\end{itemize}

% Task 3
\section{Giới thiệu và hướng dẫn chơi Game}
\subsection{Giới thiệu về trò chơi rắn săn mồi (Snake Game)}
Snake là một thể loại trò chơi hành động trong đó người chơi điều khiển phần đầu của một đường thẳng ngày càng dài, thường được thiết kế theo hình dạng một con rắn. Người chơi phải tránh để rắn va vào chướng ngại vật hoặc chính cơ thể mình, điều này trở nên khó hơn khi rắn dài ra.
\\\\
Thể loại này bắt nguồn từ trò chơi arcade cạnh tranh Blockade năm 1976 của Gremlin Industries, trong đó mục tiêu là sống lâu hơn đối thủ. Blockade và những phiên bản sao chép đầu tiên chỉ mang tính chất trừu tượng và không sử dụng hình ảnh rắn. Ý tưởng này sau đó phát triển thành một biến thể chơi đơn, nơi một đường thẳng có phần đầu và đuôi sẽ dài ra mỗi khi ăn thức ăn - thường là táo hoặc trứng - làm tăng khả năng va chạm vào chính nó. Sự đơn giản và yêu cầu kỹ thuật thấp của các trò chơi thuộc thể loại rắn đã tạo ra hàng trăm phiên bản, một số có từ "snake" hoặc "worm" trong tên. Trò chơi arcade Tron năm 1982, dựa trên bộ phim cùng tên, bao gồm chế độ chơi "Light Cycles" với lối chơi giống rắn, và một số trò chơi rắn sau này cũng lấy cảm hứng từ chủ đề này.
\\\\
Sau khi một phiên bản đơn giản có tên Snake được cài đặt sẵn trên điện thoại Nokia vào năm 1998, thể loại trò chơi rắn đã thu hút sự quan tâm trở lại.

\subsubsection{Lịch sử ra đời}
Thể loại game Snake bắt đầu với trò chơi arcade Blockade (1976) được phát triển và phát hành bởi Gremlin Industries. Cùng năm đó, trò chơi này được sao chép thành Bigfoot Bonkers. Năm 1977, Atari, Inc. phát hành hai tựa game lấy cảm hứng từ Blockade: trò chơi arcade Dominos và trò chơi Surround cho hệ máy Atari VCS. Surround là một trong chín tựa game ra mắt cùng Atari VCS tại Mỹ và được Sears bán dưới tên gọi Chase. Cùng năm, một trò chơi tương tự được phát hành trên Bally Astrocade với tên Checkmate, và Mattel ra mắt Snafu trên hệ máy Intellivision vào năm 1982.
\\\\
Phiên bản đầu tiên trên máy tính gia đình, Worm, được Peter Trefonas lập trình cho TRS-80 và xuất bản bởi tạp chí CLOAD vào năm 1978. Sau đó, tác giả phát triển các phiên bản cho PET và Apple II. Một phiên bản chính thức của trò chơi arcade Hustle, bản sao của Blockade, được phát hành bởi Milton Bradley trên TI-99/4A vào năm 1980.
\\\\
Snake Byte (1982) dành cho Atari 8-bit, Apple II và VIC-20 có lối chơi một người, nơi rắn ăn táo để hoàn thành màn chơi và dài hơn sau mỗi lần ăn. Trong Snake trên BBC Micro (1982) của Dave Bresnen, người chơi điều khiển rắn bằng các phím mũi tên trái/phải theo hướng rắn đang đi. Trò chơi tăng tốc độ khi rắn dài hơn và chỉ có một mạng.
\\\\
Nibbler (1982) là trò chơi arcade một người, nơi rắn di chuyển nhanh và vừa khít trong mê cung, khác biệt so với nhiều thiết kế rắn khác. Một phiên bản chơi đơn khác nằm trong trò chơi arcade Tron (1982), lấy chủ đề "Light Cycles". Tron đã làm mới lại ý tưởng trò chơi rắn, và nhiều tựa game sau này đã mượn chủ đề này.
\\\\
Từ năm 1991, Nibbles được đưa vào MS-DOS như một chương trình mẫu của QBasic. Năm 1992, Rattler Race được phát hành trong gói giải trí Microsoft Entertainment Pack thứ hai, bổ sung rắn đối thủ vào lối chơi ăn táo quen thuộc.

\subsubsection{Di sản}
Vào năm 1996, tạp chí Next Generation xếp hạng trò chơi này ở vị trí thứ 41 trong danh sách "100 Trò Chơi Hay Nhất Mọi Thời Đại", nhấn mạnh sự cần thiết của cả phản xạ nhanh và sự suy tính trước. Thay vì đặt tiêu đề cho một phiên bản cụ thể, họ đã liệt kê nó là "trò chơi Rắn" trong ngoặc kép.
\\\\
Vào ngày 29 tháng 11 năm 2012, Bảo tàng Nghệ thuật Hiện đại tại thành phố New York đã thông báo rằng phiên bản trò chơi Snake trên Nokia là một trong 40 trò chơi mà các nhà giám tuyển mong muốn thêm vào bộ sưu tập của bảo tàng trong tương lai.

\subsection{Hướng dẫn chơi}
\subsubsection{Mục Tiêu Của Trò Chơi}
\begin{itemize}
    \item Điều khiển con rắn ăn càng nhiều thức ăn (thường là các điểm nhỏ trên màn hình) càng tốt.
    \item Mỗi khi ăn được thức ăn, con rắn sẽ dài thêm.
    \item Trò chơi kết thúc nếu rắn đâm vào chính nó hoặc vào tường.
\end{itemize}

\subsubsection{Cách điều khiển}
\begin{itemize}
    \item Sử dụng các phím mũi tên trên bàn phím hoặc phím điều khiển trên thiết bị di động để di chuyển rắn:
    \begin{itemize}
        \item Mũi tên lên: Đi lên.
        \item Mũi tên xuống: Đi xuống.
        \item Mũi tên trái: Quẹo trái.
        \item Mũi tên phải: Quẹo phải.
    \end{itemize}
\end{itemize}

\subsubsection{Quy tắc}
\begin{itemize}
    \item Rắn chỉ có thể di chuyển theo một hướng tại một thời điểm và không thể đi ngược lại hướng đang đi (ví dụ: nếu rắn đang đi lên thì không thể đi xuống ngay lập tức).
    \item Khi rắn ăn thức ăn, độ dài của rắn tăng lên và tốc độ di chuyển có thể tăng lên.
    \item Trò chơi kết thúc khi rắn:
    \begin{itemize}
        \item Đụng vào thân của chính nó.
        \item Đụng vào tường.
    \end{itemize}
\end{itemize}

\subsubsection{Mẹo chơi}
\begin{itemize}
    \item Lên kế hoạch di chuyển trước: Đừng chỉ tập trung vào việc ăn thức ăn mà quên mất lối thoát.
    \item Giữ khoảng cách với thân của rắn để tránh va chạm.
    \item Thử đi theo các đường vòng tròn lớn để tạo không gian di chuyển rộng rãi.
    \item Cẩn thận khi rắn dài ra, vì khả năng va chạm với thân của chính nó sẽ tăng lên.
\end{itemize}

\subsubsection{Biến thể}
\begin{itemize}
    \item Một số phiên bản game sẽ có những bổ sung như:
    \begin{itemize}
        \item Thức ăn đặc biệt: Có thể xuất hiện ngẫu nhiên và mang lại điểm số cao hơn.
        \item Tường chắn: Gây khó khăn hơn trong việc di chuyển.
        \item Chế độ không giới hạn: Khi đi qua tường, rắn sẽ xuất hiện ở phía đối diện của màn hình.
    \end{itemize}
\end{itemize}

% Task 4
\section{Tài liệu kỹ thuật của trò chơi}
\subsection{Tóm tắt}
\begin{itemize}
    \item Khi chạy chương trình, khi bảng được khởi tạo; con rắn sẽ được khởi tạo ở vị trí giữa bảng.
    \item Rắn sẽ tự di chuyển theo hướng cho trước (Hướng di chuyển: \textbf{Phải})
    \item Người chơi phải điều khiển hướng đi của của con rắn bằng các keyword tương ứng từ bàn phím: \textbf{W} - Đi lên, \textbf{A} - Qua trái, \textbf{D} - Qua phải, \textbf{S} - Đi lùi.
    \item Thức ăn được khởi tạo ở vị trí ngẫu nhiên trên bảng.
    \item Mục tiêu của trò chơi là ăn càng nhiều để có được số điểm cao nhất (\textbf{Lưu ý:} Trò chơi sẽ kết thúc nếu người chơi đụng vào tường hoặc thân).
    \item Điểm được hiển thị ở dưới cùng bảng trò chơi.
\end{itemize}

\subsection{Kỹ thuật của trò chơi}
\subsubsection{Hàm chính}
\begin{center}
    \begin{itemize}
        \item \textbf{class CSnake} bao gồm các thuộc tính và phương thức sau:
        \begin{itemize}
            \item Thuộc tính:
            \begin{itemize}
                \item length: Chiều dài của rắn.
                \item direction: Hướng di chuyển của rắn.
                \item foodCounter: Đếm số lượng thức ăn đã ăn.
                \item ground[MAX][MAX]: Bảng trò chơi.
                \item body[HEIGHT * WIDTH]: Vị trí các phần của rắn.
            \end{itemize}
        \end{itemize}
        \begin{itemize}
            \item Phương thức:
            \begin{itemize}
                \item initSnake(): Khởi tạo vị trí ban đầu của rắn ở giữa bảng.
                \item initGround(): Khởi tạo bảng trò chơi, với các bức tường xung quanh.
                \item updateFood(): Khởi tạo ăn mới tại vị trí ngẫu nhiên trên bảng.
                \item updateSnake(int delay): Cập nhật vị trí của rắn, kiểm tra va chạm và xử lý ăn thức ăn.
                \item drawBackground(): Vẽ bảng trò chơi và các đối tượng.
                \item inputReading(const char ch): Xử lý đầu vào từ người chơi.
            \end{itemize}
        \end{itemize}
    \end{itemize}
    
    \begin{lstlisting}
        class CSnake
        {
        private:
            int length;
            int direction;
            int foodCounter;
            int ground[MAX][MAX];
            Coordinate body[HEIGHT * WIDTH];
        
        public:
            void initSnake();
            void initGround();
            void updateFood();
            void updateSnake(int delay);
            void inputReading(const char ch);
            int getFoodCounter();
            void drawBackground();
        };
    \end{lstlisting}
\end{center}
\vspace{2cm}

\begin{center}
    \begin{itemize}
        \item \textbf{void snake::initSnake()} - Khởi tạo rắn, và vị trí ban đầu cho rắn.
    \end{itemize}
    
    \begin{lstlisting}
    void CSnake::initSnake()
    {
        length = SNAKE_LENGTH;
        // Snake will appear at middle of the board
        body[0].x = WIDTH / 2;
        body[0].y = HEIGHT / 2;
        direction = input;
        foodCounter = 0;
    
        for (int i = 1; i < length; i++)
        {
            body[i].x = body[i - 1].x - dx[direction];
            body[i].y = body[i - 1].y - dy[direction];
        }
    
        for (int i = 0; i < length; i++)
        {
            ground[body[i].y][body[i].x] = SNAKE;
        }
    };
    \end{lstlisting}
\end{center}

\begin{center}
    \begin{itemize}
        \item \textbf{void CSnake::initGround()} - Khởi tạo bảng trò chơi.
    \end{itemize}
    
    \begin{lstlisting}
    void CSnake::initGround()
    {
        for (i = 0; i < MAX; i++)
            for (j = 0; j < MAX; j++)
                ground[i][j] = 0;
    
        for (i = 0; i <= WIDTH + 1; i++)
        {
            ground[0][i] = WALL;
            ground[HEIGHT + 1][i] = WALL;
        }
    
        for (i = 0; i <= HEIGHT + 1; i++)
        {
            ground[i][0] = WALL;
            ground[i][WIDTH + 1] = WALL;
        }
    };
    \end{lstlisting}
\end{center}
\vspace{2cm}

\begin{center}
    \begin{itemize}
        \item \textbf{void CSnake::updateFood()} - Khởi tạo vị trí thức ăn xuất hiện ngẫu nhiên trên bảng; đảm bảo thức ăn không bị khởi tạo trên rắn hoặc tường.
    \end{itemize}
    
    \begin{lstlisting}
    void CSnake::updateFood()
    {
        int x, y;
    
        do
        {
            x = rand() % WIDTH + 1;
            y = rand() % HEIGHT + 1;
        } while (ground[y][x] != NOTHING);
    
        ground[y][x] = FOOD;
        foodCounter++;
    };
    \end{lstlisting}
\end{center}

\begin{center}
    \begin{itemize}
        \item \textbf{void CSnake::updateSnake(int delay)}
        \begin{itemize}
            \item Hàm giúp cập nhật vị trí của rắn.
            \item Kiểm tra va chạm.
            \item Xử lý ăn thức ăn.
            \item Cập nhật trạng thái trên ground.
            \item Cập nhật điểm (nếu rắn ăn thức ăn).
        \end{itemize}
    \end{itemize}
    
    \begin{lstlisting}
    void CSnake::updateSnake(int delay)
    {
        int i;
        Coordinate prev[HEIGHT * WIDTH];
    
        for (i = 0; i < length; i++)
        {
            prev[i].x = body[i].x;
            prev[i].y = body[i].y;
        }
    
        if (input != EXIT || !oppositeDirection(direction, input))
            direction = input;
    
        // Update head
        body[0].x = prev[0].x + dx[direction];
        body[0].y = prev[0].y + dy[direction];
    
        // Check if head position is not colliding with anything
        if (ground[body[0].y][body[0].x] < NOTHING)
        {
            item = -1;
            ground[HEIGHT + 2][WIDTH / 2] = GAMEOVER;
            return;
        }
    
        // If head finds a piece of food
        if (ground[body[0].y][body[0].x] == FOOD)
        {
            length++;
            item = FOOD;
        }
        else
        {
            ground[body[length - 1].y][body[length - 1].x] = NOTHING;
            item = NOTHING;
        }
    
        ground[body[0].y][body[0].x] = SNAKE;
    
        // Move the rest of the body
        for (i = 1; i < length; i++)
        {
            body[i].x = prev[i - 1].x;
            body[i].y = prev[i - 1].y;
            ground[body[i].y][body[i].x] = SNAKE;
        }
    
        ground[HEIGHT + 3][WIDTH / 2] = SCORE;
        usleep(delay);
        return;
    };
    \end{lstlisting}
\end{center}

\begin{center}
    \begin{itemize}
        \item \textbf{void CSnake::drawBackground()}
        \begin{itemize}
            \item Xóa màn hình.
            \item Duyệt qua từng ô trong ground để vẽ:
            \begin{itemize}
                \item Tường.
                \item Thân rắn, đầu rắn.
                \item Thức ăn.
                \item Kết thúc game hoặc điểm số.
            \end{itemize}
        \end{itemize}
    \end{itemize}
    
    \begin{lstlisting}
    void CSnake::drawBackground()
    {
        clear_background();
    
        for (i = 0; i <= HEIGHT + 4; i++)
        {
            for (j = 0; j <= WIDTH + 4; j++)
            {
                switch (ground[i][j])
                {
                case NOTHING:
                    cout << " ";
                    break;
                case WALL:
                    if ((i == 0 && j == 0) || (i == 0 && j == WIDTH + 1) || (i == HEIGHT + 1 && j == 0) || (i == HEIGHT + 1 && j == WIDTH + 1))
                    {
                        cout << "+";
                    }
                    else if (j == 0 || j == HEIGHT + 1)
                    {
                        cout << "|";
                    }
                    else
                    {
                        cout << "-";
                    }
                    break;
                case SNAKE:
                    if (i == body[0].y && j == body[0].x)
                    {
                        cout << "#";
                    }
                    else
                    {
                        cout << "+";
                    }
                    break;
                case FOOD:
                    cout << "%";
                    break;
                case GAMEOVER:
                    cout << "GAME OVER!";
                    break;
                case SCORE:
                    cout << "Score: " << (foodCounter - 1);
                    break;
                }
            }
    
            cout << endl;
        }
    };
    \end{lstlisting}
\end{center}

\begin{center}
    \begin{itemize}
        \item \textbf{void CSnake::inputReading(const char ch)} - Hàm đọc phím điều khiển từ người chơi.
    \end{itemize}
    
    \begin{lstlisting}
    // Snake movement
    void CSnake::inputReading(const char ch)
    {
        if (ch == 'd')
        {
            input = RIGHT;
        }
        if (ch == 'w')
        {
            input = UP;
        }
        if (ch == 'a')
        {
            input = LEFT;
        }
        if (ch == 's')
        {
            input = DOWN;
        }
    };
    \end{lstlisting}
\end{center}

\subsubsection{Hàm main()}
\begin{center}
    \begin{itemize}
        \item Khởi tạo game:
        \begin{itemize}
            \item Gọi và khởi tạo đối tượng \textbf{snake}.
            \item Gọi các hàm \textbf{initGround()}, \textbf{initSnake()}, \textbf{updateFood()}, và \textbf{drawBackground()}.
        \end{itemize}
        
        \item Vòng lặp game:
        \begin{itemize}
            \item Kiểm tra phím bấm bằng \textbf{\_kbhit()}. Nếu có, đọc phím (\textbf{cin >> ch}) và chuyển hướng.
            \item Cập nhật trạng thái rắn (\textbf{updateSnake()}).
            \item Vẽ bảng chơi (\textbf{drawBackground()}).
            \item Nếu rắn ăn thức ăn, tạo thêm thức ăn mới (\textbf{updateFood()}).
        \end{itemize}
        
        \item Kết thúc:
        \begin{itemize}
            \item Khi rắn đụng tường hoặc thân, thoát khỏi vòng lặp và kết thúc chương trình.
        \end{itemize}
    \end{itemize}

    \begin{lstlisting}
        int main()
    {
        char ch;
        microseconds = 100000;
        srand(time(NULL));
    
        CSnake snake;
        snake.initGround();
        snake.initSnake();
        snake.updateFood();
        snake.drawBackground();
    
        do
        {
            if (_kbhit())
            {
                cin >> ch;
                snake.inputReading(ch);
            }
    
            snake.updateSnake(microseconds);
            snake.drawBackground();
    
            if (item == FOOD)
            {
                snake.updateFood();
            }
        } while (item >= 0 && input != EXIT);
    
        snake.drawBackground();
        return 0;
    }
    \end{lstlisting}
\end{center}

\subsubsection{Hàm hỗ trợ}
\begin{center}
    \begin{itemize}
        \item Hàm giúp di chuyển con trỏ đến vị trí (x, y) trong terminal.
    \end{itemize}
    
    \begin{lstlisting}
        void gotoxy(int x, int y)
        {
            move(y, x);
        }
    \end{lstlisting}
\end{center}

\begin{center}
    \begin{itemize}
        \item Hàm giúp kiểm tra xem có phím nào được nhấn không.
    \end{itemize}
    
    \begin{lstlisting}
        // http://www.flipcode.com/archives/_kbhit_for_Linux.shtml
        int _kbhit()
        {
            static const int STDIN = 0;
            static bool initialized = false;
        
            if (!initialized)
            {
                // Use termios to turn off line buffering
                termios term;
                tcgetattr(STDIN, &term);
                term.c_lflag &= ~ICANON;
                tcsetattr(STDIN, TCSANOW, &term);
                setbuf(stdin, NULL);
                initialized = true;
            }
        
            // int bytesWaiting;
            ioctl(STDIN, FIONREAD, &bytesWaiting);
            return bytesWaiting;
        }
    \end{lstlisting}
\end{center}

\begin{center}
    \begin{itemize}
        \item Hàm giúp xóa màn hình terminal.
    \end{itemize}
    
    \begin{lstlisting}
        void clear_background(void)
        {
            system(clearcommand);
        }
    \end{lstlisting}
\end{center}

\begin{center}
    \begin{itemize}
        \item Hàm giúp kiểm tra xem hướng di chuyển có đối lập không.
    \end{itemize}
    
    \begin{lstlisting}
        int oppositeDirection(int input1, int input2)
        {
            if (input1 == RIGHT && input2 == LEFT)
            {
                return 1;
            }
            if (input1 == LEFT && input2 == RIGHT)
            {
                return 1;
            }
            if (input1 == UP && input2 == DOWN)
            {
                return 1;
            }
            if (input1 == DOWN && input2 == UP)
            {
                return 1;
            }
        
            return 0;
        }
    \end{lstlisting}
\end{center}

% Task 5, 6, 7 - chua lam
\section{Quá trình làm việc nhóm}
% Task 5 - Mô tả quá trình làm việc nhóm : Dựa trên diễn biến công việc đã được nhóm thực hiện thực tế trong Trello hoặc Git.. nhóm SV trình bày quá trình làm việc của cả nhóm, các vấn đề đã xảy ra và nhóm đã giải quyết như thế nào. Phần này tối thiểu 2 trang, tối đa 5 trang.

% Task 6 - Các kỹ năng mà nhóm SV áp dụng khi tham gia đồ án này. Phần này tối thiểu 1 trang, tối đa 3 trang.
\subsection{Kỹ năng đạt được}

% Task 7 - Đánh giá việc thực hiện hợp đồng nhóm
\subsection{Đánh giá hợp đồng nhóm}
\begin{tabular}{|p{2cm}|p{3cm}|p{3cm}|p{3cm}|p{3cm}|}
    \hline
    \textbf{Tiêu chí} & \textbf{4 - Xuất sắc} & \textbf{3 - Tốt} & \textbf{2 - Trung bình} & \textbf{1 - Kém}\\
    \hline
    \textbf{Thái độ} & Không bao giờ chỉ trích công khai dự án hoặc công việc của người khác. Luôn có thái độ tích cực. & Hiếm khi chỉ trích công khai dự án hoặc công việc của người khác. & Thỉnh thoảng chỉ trích công khai dự án hoặc công việc của các thành viên khác trong nhóm. & Thường xuyên chỉ trích công khai dự án hoặc công việc của các thành viên khác trong nhóm.\\
    \hline
    \textbf{Đóng góp} & Thường xuyên cung cấp ý tưởng hữu ích khi tham gia vào nhóm và thảo luận trong lớp. Là một người lãnh đạo rõ ràng và đóng góp nhiều nỗ lực. & Thường cung cấp ý tưởng hữu ích khi tham gia vào nhóm và thảo luận trong lớp. Là một thành viên nhóm mạnh và cố gắng hết sức. & Thỉnh thoảng cung cấp ý tưởng hữu ích khi tham gia vào nhóm và thảo luận trong lớp. Làm những gì được yêu cầu. & Hiếm khi cung cấp ý tưởng hữu ích khi tham gia vào nhóm và thảo luận trong lớp. Có thể từ chối tham gia.\\
    \hline
    \textbf{Tinh thần trách nhiệm} & Sẵn lòng chấp nhận và hoàn thành vai trò cá nhân trong nhóm. & Chấp nhận và hoàn thành vai trò cá nhân trong nhóm. & Đóng góp cho nhóm khi được nhắc nhở. & Chỉ đóng góp cho nhóm khi được nhắc nhở.\\
    \hline
    \textbf{Giải quyết vấn đề} & Chủ động tìm kiếm và đề xuất các giải pháp cho vấn đề. & Cải thiện các giải pháp được đề xuất bởi người khác. & Không đề xuất hoặc cải thiện giải pháp, nhưng sẵn lòng thử các giải pháp do người khác đề xuất. & Không cố gắng giải quyết vấn đề hoặc giúp người khác giải quyết vấn đề. Để người khác làm công việc.\\
    \hline
    \textbf{Chất lượng công việc} & Cung cấp công việc với chất lượng cao nhất. & Cung cấp công việc chất lượng cao. & Cung cấp công việc thỉnh thoảng cần được kiểm tra/ làm lại bởi các thành viên khác trong nhóm để đảm bảo chất lượng. & Cung cấp công việc thường cần được kiểm tra/làm lại bởi người khác để đảm bảo chất lượng.\\
    \hline
\end{tabular}

\subsubsection{Đánh giá từng thành viên}

\subsubsection{Chỉ tiêu đánh giá}
\begin{tabular}{p{5cm} p{5cm} p{5cm}}
    \textbf{Member 1} & \textbf{Member 2} & \textbf{Member 3} \\
    \begin{tabular}{@{}c@{}}
        Trần Mai Uyên Nhi \\
        % \includegraphics[width=5cm]{nhi.jpg}
    \end{tabular}
    &
    \begin{tabular}{@{}c@{}}
        Phạm Tiến Anh \\
        % \includegraphics[width=5cm]{signature.png}
    \end{tabular}
    &
    \begin{tabular}{@{}c@{}}
        Nguyễn Thanh Minh \\
        % \includegraphics[width=5cm]{signature.png}
    \end{tabular}
    
\end{tabular}
\\\\
\begin{tabular}{p{5cm} p{5cm} p{5cm}}
    \textbf{Member 4} & \textbf{Member 5} & \textbf{} \\
    \begin{tabular}{@{}c@{}}
        Lưu Đinh Đại Đức \\
        % \includegraphics[width=5cm]{nhi.jpg}
    \end{tabular}
    &
    \begin{tabular}{@{}c@{}}
        Trương Anh Vũ \\
        % \includegraphics[width=5cm]{signature.png}
    \end{tabular}
\end{tabular}

% % ...
% % \section{Tài Liệu Kỹ Thuật của Trò Chơi:}
% % \subsection{Công nghệ sử dụng}
% % \begin{itemize}
% %     \item \textbf{React}: React được chọn làm nền tảng cho giao diện người dùng do khả năng xây dựng ứng dụng đơn trang (SPA) mạnh mẽ, giúp quản lý trạng thái và các thành phần UI một cách hiệu quả. Với cấu trúc component-based, React giúp chia các phần UI thành các khối nhỏ và tái sử dụng, từ đó tăng tính linh hoạt và dễ bảo trì cho dự án.

% %     \item \textbf{Phaser}: Phaser là một thư viện game 2D mạnh mẽ và phù hợp cho các dự án trò chơi nhẹ nhàng như trò rắn săn mồi. Nó cung cấp các công cụ giúp quản lý các yếu tố trong trò chơi, từ các hoạt cảnh, xử lý va chạm, cho đến việc thiết lập các hiệu ứng động. Kết hợp với React, Phaser giúp duy trì tính tương tác động trong khi vẫn đảm bảo hiệu suất tốt và dễ quản lý.

% %     \item \textbf{Thành phần trong trò chơi}

% %     \item \textbf{Game}: `Game` là đối tượng cốt lõi của Phaser, dùng để khởi tạo toàn bộ trò chơi. Nó chịu trách nhiệm thiết lập cấu hình ban đầu (kích thước, chế độ hiển thị, và các tùy chọn khác), tạo ra môi trường để các scene hoạt động. Game là trung tâm điều khiển các thành phần khác và xử lý vòng lặp game liên tục.

% %     \item \textbf{Scene}: Scene là môi trường nơi trò chơi diễn ra, đóng vai trò tổ chức các đối tượng và logic trong mỗi phần của trò chơi (ví dụ: màn chơi, menu, hoặc màn hình kết thúc). Scene giúp phân tách các trạng thái khác nhau của trò chơi, cho phép chuyển đổi dễ dàng giữa chúng mà không ảnh hưởng đến các phần khác. Mỗi scene có thể quản lý hoạt cảnh, tương tác, và các sự kiện riêng.

% %     \item \textbf{Camera}: Camera giúp xác định khu vực của scene mà người chơi sẽ thấy. Trong các trò chơi lớn, camera có thể di chuyển để theo dõi các đối tượng chính hoặc thu phóng để tạo ra những hiệu ứng thị giác. Camera có thể điều chỉnh để tạo ra các trải nghiệm khác nhau, như tập trung vào các đối tượng nhất định hoặc cho phép người chơi xem toàn bộ scene.

% %     \item \textbf{Physics}: Đây là hệ thống quản lý va chạm và các quy luật vật lý trong trò chơi. Phaser cung cấp nhiều loại vật lý, như Arcade Physics, giúp tạo ra các tương tác đơn giản và hiệu quả giữa các đối tượng. Physics đảm bảo các đối tượng di chuyển, va chạm, và phản ứng theo các quy tắc mà trò chơi yêu cầu.

% %     \item \textbf{Input}: Input quản lý các sự kiện đầu vào từ người chơi, như bàn phím, chuột hoặc cảm ứng. Input giúp trò chơi xử lý hành động của người chơi và phản hồi lại, như di chuyển nhân vật, kích hoạt chức năng, hoặc điều khiển menu.
% % \end{itemize}

% % \subsection{Sơ lược về các chức năng chính:}
% % \begin{itemize}
% %     \item \textbf{Di chuyển và kiểm soát rắn}: Người chơi có thể điều khiển rắn theo các hướng lên, xuống, trái và phải. Hướng di chuyển của rắn được xác định dựa trên đầu vào của bàn phím.

% %     \item \textbf{Hệ thống va chạm:} Trò chơi có chức năng phát hiện va chạm giữa đầu rắn và thân rắn, giúp trò chơi kết thúc nếu rắn đâm vào chính mình.

% %     \item \textbf{Hệ thống tạo thức ăn:} Thức ăn sẽ xuất hiện ở vị trí ngẫu nhiên trong khu vực chơi. Khi rắn ăn được thức ăn, chiều dài của nó sẽ tăng lên, và thức ăn mới sẽ được sinh ra ở vị trí khác.

% %     \item \textbf{Hiệu ứng animations cho rắn:} Các đoạn của rắn bao gồm đầu, thân, và đuôi, đều có các hiệu ứng animations riêng, tạo cảm giác mượt mà khi rắn di chuyển.
% % \end{itemize}

% % \subsection{Các lớp và cấu trúc chính của chương trình:}
% % \begin{itemize}
% %     \item \textbf{Lớp Snake:}
% % \end{itemize}

% % Đây là lớp chính quản lý các tính năng của con rắn, từ điều khiển, di chuyển, đến phát hiện va chạm.

% % Lớp này chứa thông tin về các đoạn của rắn, hướng di chuyển hiện tại và các trạng thái khác của trò chơi.

% % \begin{itemize}
% %     \item \textbf{Interface SnakePart:}
% % \end{itemize}

% % Đại diện cho từng đoạn của rắn với vị trí và animations.

% % Mỗi phần của rắn bao gồm một sprite và vị trí hiện tại của nó.

% % \begin{itemize}
% %     \item \textbf{Type Direction và BodyAnimKeys:}
% % \end{itemize}

% % \textbf{Direction:} Quy định hướng di chuyển của rắn.

% % \textbf{BodyAnimKeys:} Quy định animations dành cho phần thân của rắn.

% % \begin{itemize}
% %     \item \textbf{Phương thức move():}
% % \end{itemize}

% % Phương thức này giúp rắn di chuyển theo hướng hiện tại.

% % Mỗi khi \textbf{move()} được gọi, phương thức tính toán vị trí mới của đầu rắn và di chuyển các đoạn thân theo sau để tạo ra hiệu ứng rắn di chuyển mượt mà.

% % \begin{itemize}
% %     \item \textbf{Phương thức grow():}
% % \end{itemize}

% % Được gọi khi rắn ăn được thức ăn. Phương thức này thêm một đoạn mới vào cuối rắn, làm tăng chiều dài của rắn.

% % \begin{itemize}
% %     \item \textbf{Hệ thống animations (createAnimations):}
% % \end{itemize}

% % Lớp này sử dụng các hình ảnh để tạo animations cho từng đoạn của rắn và có các animations khác nhau cho đầu rắn khi rắn di chuyển theo các hướng.

% % \begin{itemize}
% %     \item \textbf{Phương thức initializeSnake():}
% % \end{itemize}

% % Phương thức khởi tạo các đoạn của rắn và đặt rắn ở vị trí ban đầu trên màn hình.

% % \subsection{Giải thuật và cấu trúc dữ liệu:}
% % Dữ liệu của rắn được lưu trữ dưới dạng mảng snakeParts, trong đó mỗi phần tử là một SnakePart, biểu diễn một đoạn của rắn.\\

% % \textbf{Giải thuật điều khiển di chuyển:} Dựa trên hướng đi của người chơi, chương trình tính toán vị trí mới cho rắn và di chuyển các đoạn thân theo hướng đầu rắn để tạo hiệu ứng di chuyển.\\

% % \textbf{Giải thuật sinh thức ăn ngẫu nhiên:} Tọa độ của thức ăn được sinh ra ngẫu nhiên trên màn hình với các giá trị chia hết cho kích thước ô lưới (grid size), giúp nó nằm gọn trong các ô của màn hình.
\end{document}
